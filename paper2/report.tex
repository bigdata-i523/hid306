\documentclass[sigconf]{acmart}

\input{format/i523}

\begin{document}
	\title{Why Deep Learning matters in IoT Data Analytics?}
	
	
	\author{Murali Cheruvu}
	\orcid{xxxx-xxxx-xxxx}
	\affiliation{%
		\institution{Indiana University}
		\streetaddress{3209 E 10th St}
		\city{Bloomington} 
		\state{Indiana} 
		\postcode{47408}
	}
	\email{mcheruvu@iu.edu}
	
	% The default list of authors is too long for headers}
	\renewcommand{\shortauthors}{M. Cheruvu}
	
	
	\begin{abstract}
		
	The Deep Learning is unique in machine learning algorithms to analyze supervised and unsupervised datasets. Big Data challenges like high volumes, multi-dimensionality and feature engineering are well addressed using Deep Learning algorithms. Deep Leaning, with Edge and distributed Mesh computing, is best suited to handle IoT Analytics of millions of sensors producing petabytes of time-series data.
		
	\end{abstract}
	
	\keywords{i523, hid306, IoT, Deep Learning, Big Data Analytics}
	
	\maketitle
	

	
	\section{Introduction}		

	Historically supervised machine learning algorithms: linear regression, decision trees, Support Vector Machine (SVM), neural networks, etc. are popular in analyzing the datasets or in predicting future trends based sample training data. K-Clustering algorithms are good for unsupervised datasets to categorize based on the identified patterns in the data. While there are so many factors - nature of the domain, sample size of the dataset and number of attributes defining characteristics of the data - decide which machine learning algorithm works better, Deep Learning algorithms are getting greater traction. They address challenging analytics tasks, such as high-dimensionality and automatic creation of new features from existing complex hierarchical features, very well. 
		
	\section{Neural Networks}
	
	\section{Deep Learning}
	\subsection{Curse of Dimensionality}
	\subsection{Feature Engineering}
	\subsection{Types of Deep Neural Networks}
	
	\section{IoT Data Analytics}
	\subsection{Complexity}
	\subsection{Scalability}
	
	\section{Conclusion}		

	In contrast to traditional machine learning solutions, Deep Learning not only scales well with high volumes of input data but also facilitates in automatic decomposition of complex data representations of unsupervised and uncategorized data. Automatic discovery of new features, from convolutional or recurrent neural networks, makes Deep Learning predominant among all machine learning algorithms. Fuzzy logic of Deep Learning algorithms make them difficult to understand perhaps more adoption helps getting better handle at them. Deep Learning algorithms need deep research in validating the process of challenging Big Data Analytics tasks such as data tagging, scalability, semantic learning and the reliability of the predictions. 
 
	
	\begin{acks}		
	
		The author would like to thank Dr. Gregor von Laszewski and the Teaching Assistants for their support and valuable suggestions.
		
	\end{acks}

	\bibliographystyle{ACM-Reference-Format}
	\bibliography{report} 
	

	
\end{document}
